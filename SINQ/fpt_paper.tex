\documentclass[12pt,letterpaper]{article}

\usepackage[letterpaper]{geometry}
\geometry{top=1.0in, bottom=1.0in, left=1.0in, right=1.0in}

\usepackage{setspace}
\doublespacing

\usepackage{fancyhdr}
\pagestyle{fancy}
\lhead{}
\chead{}
\rhead{Haynes \thepage}
\lfoot{}
\cfoot{}
\rfoot{}
\renewcommand{\headrulewidth}{0pt}
\renewcommand{\footrulewidth}{0pt}

\begin{document}
\begin{flushleft}

Gregory Haynes \\
Parms \\
Freedom, Privacy, and Technology \\
12/16/2009 \\
\\
\begin{center}Internet Surveillance in Germany\end{center}

\setlength{\parindent}{0.5in}

When you first hear the word “surveillance,” you may think of old-fashioned
methods of surveillance, conjuring images in your head of a police officer
eavesdropping through a window or wiretapping a phone line. Due to the
widespread use and inherent nature of the Internet, however, this is no
longer how surveillance has to work. When data travels between computers on
the Internet, the data passes through many machines, each machine being a
possible surveillance point in the data's line of transit. The Internet can
be thought of as similar to the children's game “Telephone,” where a child
whispers a message to his or her neighbor, who then proceeds to pass the
message on to his or her neighbor, and so on. Because the Internet is
designed in this way, governments invested in wielding greater control and
security may not face barriers in using the technology to watch activity and
censor information.

Internet surveillance techniques entered the public arena in 1993 with the
introduction of the “Digital Telephony and Privacy Improvement Act” in the
United States, a bill that would force network providers to give up user
e-mail documents upon request. Internet surveillance technology rapidly
improved and in the year 2000, the FBI announced its creation of a
packet-sniffing program named Carnivore. This program was to be installed at
Internet Service Providers across the United States, and perhaps the world,
in order to efficiently monitor a specified user’s complete Internet
activity. Carnivore was eventually abandoned for the more flexible
mass-Internet surveillance system, NarusInsight.

Governments primarily use Internet surveillance to aid in the prevention of
terrorism and other crimes. The range of interests in Internet surveillance
is vast, however, ranging from parents who wish to watch their child's on
line habits to employees who wish to prevent employees from e-slacking. With
all of these techniques, though, there remains a common problem in Internet
surveillance: finding useful information among the massive amount of data
collected. Most systems still rely on key words to find what they are
looking for, a process known as deep packet inspection, but this is quickly
changing as technology improves.

One of the first laws created regarding electronic surveillance was the
United States' Federal Wiretap Law enacted in 1968. Since then, the U.S.
Congress has tried to keep the law up to date with progressing technologies,
a very difficult task for the slow moving law makers in Congress. As a
consequence, governments around the world are finding ways to subvert or go
around the existing laws to watch their citizens. Countries like China and
Sri Lanka use newer Internet surveillance technology to monitor their
citizens for political reasons, technology that doesn't technically violate
any laws.

The European Union has in the past been a beacon of civil liberties and
freedoms. However, in recent times the EU has become more and more like the
U.S. in how it handles domestic issues such as terrorism. Most of the EU’s
Internet technology policies are based on policies in the UK, which in turn,
carefully watches changing policies in the U.S. There is a legitimate
concern that once the harsh and privacy-bending policies of the US are
implemented in the UK that they will find their way into the EU.

Internet surveillance has always been a controversial issue. Proposals from
the government for Internet surveillance have often been received negatively
by others within the government. For example, the Defense Advanced Research
Projects Agency (DARPA) of 2002 proposed a project called Total Information
Awareness (TIA), which could allow government unrestricted and complete
access to activity on the Internet. Almost immediately, the U.S. Senate
voted to restrict DARPA's abilities. Most Americans oppose the government
surreptitiously surveilling their Internet habits. In retaliation,
organizations are attempting to thwart the government from recording private
information. The government is expanding their technology to access more
activity on the Internet, all whilst trying to keep its citizens in the
dark. The Electronic Frontier Foundation (EFF) has developed the FOIA
Litigation for Accountable Government (FLAG) project in an effort to expose
the government’s techniques to invade the general public's privacy.

In areas of the world where people are governed by oppressive regimes,
Internet privacy is a must for its citizens. The Internet is key tool for
organizing political protests, and have aided most recently with protests in
Iran and China. In these countries, government surveillance of the Internet
is not just a nuisance, but is dangerous. The repressed people thrive on the
anonymity that the Internet provides in order to have their voice heard.
They cannot achieve this with their government spying on them.

As technology progresses, Internet surveillance will only be made easier.
Packet sniffing systems utilized by governments around the world, such as
NarusInsight, may be able to collect a more complete picture of Internet
activity than is currently possible. Currently, data intercepted by
NarusInsight is based on keywords and user accounts specified by law
enforcement. As the technology and monitoring infrastructure improves,
governments may be able to collect and analyze all Internet activity in
real-time, creating profiles of Internet users based on long-term habits.
Movements are also in the works to mine and organize the latest data on
disease outbreaks around the world to create maps displaying the “global
health” and locations that should be quarantined. As surveillance technology
has improved, so has the amount of personal information that we share on
line. These hotbeds of data on networks like Facebook can potentially be
used by governments in the future. While Skype and other VoIP calls are also
relatively private at the moment thanks to heavy encryption, governments are
pushing for back doors to VoIP that would allow them to easily record on
line conversations.

German basic law contains provisions for basic privacy protections.  This is
stated in Article 10 as:
\begin{quote}
(1) Privacy of letters, posts, and telecommunications shall be inviolable. (2) Restrictions may only be ordered pursuant to a statute. Where a restriction serves to protect the free democratic basic order or the existence or security of the Federation, the statute may stipulate that the person affected shall not be informed of such restriction and that recourse to the courts shall be replaced by a review of the case by bodies and auxiliary bodies appointed by Parliament.
\end{quote}
Loosely interpretated, this means citizens have a right to privacy of data, but there are exceptions which can be made for cases of national security and warrants.

\end{document}
